\documentclass[sigconf]{acmart} % anonymous
%\usepackage[brazil]{babel}
\usepackage[utf8x]{inputenc}
\usepackage[T1]{fontenc}
\usepackage{booktabs} % For formal tables

\usepackage{tabularx} % Fitting table
\usepackage{graphicx} % Fitting table
\usepackage{flushend} % To balance the last page
%\usepackage{scalerel} % Reputation Flows figure

% Copyright
\setcopyright{none}
%\setcopyright{acmlicensed}
%\setcopyright{acmcopyright}
%%\setcopyright{rightsretained}
%\setcopyright{usgov}
%\setcopyright{usgovmixed}
%\setcopyright{cagov}
%\setcopyright{cagovmixed}

% DOI
% \acmDOI{10.475/123_4}

% ISBN
%\acmISBN{123-4567-24-567/08/06}

% Conference
%\acmConference[ICTIR'17]{ACM International Conference on the Theory of Information Retrieval}{October 2017}{Amsterdam, Netherlands} 
%\acmYear{2017}
%\copyrightyear{2017}
%
%\acmPrice{15.00}    

\begin{document}

% \vspace{9.5ex} for authors space
\title{Exploiting Structural Patterns for Health Search} 
\subtitle{Research Track - Project Proposal}

\author{Alberto Ueda}
\affiliation{%
  \institution{CS Dept, UFMG}
  %\city{Belo Horizonte} 
  %\country{Brazil} 
}
%\email{ueda@dcc.ufmg.br}

%\author{Rodrygo L. T. Santos}
%\affiliation{%
%  \institution{CS Dept, UFMG}
%  \city{Belo Horizonte} 
%  \country{Brazil} 
%}
%\email{rodrygo@dcc.ufmg.br}

% If the default list of authors is too long for headers}
% \renewcommand{\shortauthors}{A. Ueda, M. Dias, B. Ribeiro-Neto, N. Ziviani, and E. de Souza e Silva}

% To anonymize:
% \begin{anonsuppress} \end{anonsuppress}

\begin{abstract}
Health related topics have become a common theme within Information Retrieval (IR). 
A number of premiere IR publication venues, including SIGIR, have dedicated workshops, tutorials or tracks dedicated to health search. 
%Similarly, other venues such as WSDM, WWW, KDD, and ACL have all hosted health related tutorials or workshops. This shows the rising interest from the research community in contributing to health search, an area with arguably significant social impact. 
%
For instance, the vast literature available for precision medicine can make it difficult to find the most appropriate treatment for the clinician's current patient. The ability to quickly locate relevant information for a current patient using IR can be an important tool for helping clinicians find the most up-to-date evidence-based treatment for their patients.
%
In this project, we aim to propose novel strategies for clinical decision support by applying IR techniques such as similarity-based rankings, machine learning and feature engineering.
\end{abstract}

% The code below should be generated by the tool at
% http://dl.acm.org/ccs.cfm
%%%%%%%%%%%%%%%%%%%%%%%%%%%%%%%%%%%%%%%%%%%%%%%%%%%%%%%%%%%%%%%
%\begin{CCSXML}
%<ccs2012>
%<concept>
%<concept_id>10002951.10003317.10003338</concept_id>
%<concept_desc>Information systems~Retrieval models and ranking</concept_desc>
%<concept_significance>500</concept_significance>
%</concept>
%</ccs2012>
%\end{CCSXML}
%
%\ccsdesc[500]{Information systems~Retrieval models and ranking}
%%%%%%%%%%%%%%%%%%%%%%%%%%%%%%%%%%%%%%%%%%%%%%%%%%%%%%%%%%%%%%%

%TODO
%\keywords{Health Search}

\maketitle

\section{The Problem}
The problem we tackle in this project is defined by the TREC 2015 Clin­i­cal De­ci­sion Sup­port Track. As in previous editions, the main goal of this track is the re­trieval of bio­med­ical ar­ti­cles rel­e­vant for an­swer­ing generic clin­i­cal ques­tions about med­ical records~\cite{zuccon17tutorial}.
%
More specifically, the par­tic­i­pants of the track were chal­lenged with re­triev­ing for a given case re­port full-text bio­med­ical ar­ti­cles that an­swer ques­tions re­lated to sev­eral types of clin­i­cal in­for­ma­tion needs. Re­trieved ar­ti­cles are judged rel­e­vant if they pro­vide in­for­ma­tion of the spec­i­fied type that is per­ti­nent to the given case. 
%

A case re­port typ­i­cally de­scribes a chal­leng­ing med­ical case, and it is often or­ga­nized as a well-formed nar­ra­tive sum­ma­riz­ing the por­tions of a pa­tient's med­ical record that are per­ti­nent to the case. 
%
It provides in­for­ma­tion such as a pa­tient's med­ical his­tory, the pa­tient's cur­rent symp­toms, tests per­formed by a physi­cian to di­ag­nose the pa­tient's con­di­tion, the pa­tient's even­tual di­ag­no­sis, and fi­nally, the steps taken by a physi­cian to treat the pa­tient.
%
%The tar­get doc­u­ment col­lec­tion for the track is a snapshot of the Open Ac­cess Sub­set of PubMed Cen­tral (PMC), an on­line dig­i­tal data­base of freely avail­able full-text bio­med­ical lit­er­a­ture. This snapshot contains a total of 733,138 ar­ti­cles.
%
%The case reports are an­no­tated ac­cord­ing to three com­mon generic clin­i­cal ques­tion types: \textit{diagnosis, test,} or \textit{treatment}. For ex­am­ple, 
%for a case re­port la­beled "di­ag­no­sis" par­tic­i­pants should re­trieve PMC ar­ti­cles a physi­cian would find use­ful for de­ter­min­ing the di­ag­no­sis of the pa­tient de­scribed in the re­port. Sim­i­larly, 
%
%for a case re­port la­beled \textit{treat­ment}, par­tic­i­pants should re­trieve ar­ti­cles that sug­gest to a physi­cian the best treat­ment plan for the con­di­tion ex­hib­ited by the pa­tient de­scribed in the re­port. 
%
%For "test" case re­ports par­tic­i­pants should re­trieve ar­ti­cles that sug­gest rel­e­vant in­ter­ven­tions that a physi­cian might un­der­take in di­ag­nos­ing the pa­tient. 
The table \ref{tab:freq} shows an ex­am­ple of the kind of case-based top­ic used in the track.

% In ad­di­tion to an­no­tat­ing the top­ics ac­cord­ing to the type of clin­i­cal in­for­ma­tion re­quired, we are also pro­vid­ing two ver­sions of the case nar­ra­tives. The topic "de­scrip­tions" con­tain a com­plete ac­count of the pa­tients' vis­its, in­clud­ing de­tails such as their vital sta­tis­tics, drug dosages, etc., whereas the topic "sum­maries" are sim­pli­fied ver­sions of the nar­ra­tives that con­tain less ir­rel­e­vant in­for­ma­tion. 
%The document identifiers listed in the last col­umn are rel­e­vant for the given cases be­cause they can as­sist a physi­cian in de­ter­min­ing the pa­tient's di­ag­no­sis or treat­ment.


%\section{Previous Approaches}
In the previous editions of TREC, several ranking models for medicine document retrieval were proposed~\cite{roberts15}. Techniques from IR and NLP were employed by the most successful participant teams, including 
%
disease-centered document clustering and semantic word vectors using word embeddings~\cite{cbnu16cds},
%
query expansion by identification of clinical intent types, negation-aware ranking models, and synonyms~\cite{eth16cds}, and
%
supervised learning-to-rank approaches based on document similarity~\cite{merck16cds}.

\begin{table}[h]
  \scriptsize
  \caption{Example of input and expected output for the task.}
  \label{tab:freq}
  \begin{tabular}{p{6cm}p{1.5cm}}%{\textwidth}
    \toprule
    Case Report	Example &	Relevant Articles\\
    \midrule
%1	&	Type: Diagnosis. 
%
%De­scrip­tion: A 26-year-old obese woman with a his­tory of bipo­lar dis­or­der com­plains that her re­cent strug­gles with her weight and eat­ing have caused her to feel de­pressed. She states that she has re­cently had dif­fi­culty sleep­ing and feels ex­ces­sively anx­ious and ag­i­tated. She also states that she has had thoughts of sui­cide. She often finds her­self fid­gety and un­able to sit still for ex­tended pe­ri­ods of time. Her fam­ily tells her that she is in­creas­ingly ir­ri­ta­ble. Her cur­rent med­ica­tions in­clude lithium car­bon­ate and zolpi­dem. & 1087494
%
%1434505
%
%2031887 \\

Type: Treatment. 

De­scrip­tion: A 21-year-old fe­male is eval­u­ated for pro­gres­sive arthral­gias and malaise. On ex­am­i­na­tion she is found to have alope­cia, a rash mainly dis­trib­uted on the bridge of her nose and her cheeks, a del­i­cate non-pal­pa­ble pur­pura on her calves, and swelling and ten­der­ness of her wrists and an­kles. Her lab shows nor­mo­cytic ane­mia, throm­bo­cy­tope­nia, a 4/4 pos­i­tive ANA and anti-ds­DNA. Her urine is pos­i­tive for pro­tein and RBC casts. & 1065341

1459118

1526641 \\
 \bottomrule
\end{tabular}
\end{table}

\section{Our Approach}
In some contexts, finding highly related documents is a desirable feature for IR systems. For instance, given a relevant medicine article with a treatment for a patient's disease, one can be interested in finding as many documents similar to this article as possible.

For such tasks, we can apply document clustering techniques. In particular, using disease-centered document clusters improve clinical document retrieval effectiveness~\cite{cbnu16cds}. Our proposal is to still consider the disease but on a treatment-centered network. Given a patient with a specific disease and a reference treatment, we aim to provide articles with similar treatments or clinical trials. To determine the reference treatment, we aim to use learning to rank methods based on the clinical ground-truths.

\section{Evaluation Procedure} 
We aim to use both the data and the evaluation procedures provided by previous tracks of the TREC.\footnote{\url{http://trec-cds.org/2015.html}} The high­est ranked ar­ti­cles for each topic were pooled and judged by med­ical li­brar­i­ans and physi­cians from the De­part­ment of Med­ical In­for­mat­ics of the Ore­gon Health and Sci­ence Uni­ver­sity. The ar­ti­cles were judged as ei­ther \textit{def­i­nitely rel­e­vant}, \textit{def­i­nitely not rel­e­vant}, or \textit{po­ten­tially rel­e­vant}. 
%for an­swer­ing ques­tions of the spec­i­fied type about the given case re­port, 
%(the ar­ti­cle is not im­me­di­ately in­for­ma­tive on its own, but it may be rel­e­vant in the con­text of a broader lit­er­a­ture re­view). 
To obtain a graded rel­e­vance scale, the per­for­mance of the re­trieval sub­mis­sions were mea­sured using NDCG. As the previous participants' results are also available, we will be able to directly compare the effectiveness of our approaches against the top tier participants.

%\section*{ACKNOWLEDGEMENTS}
%Omitted for blind review.

\bibliographystyle{ACM-Reference-Format}
\bibliography{project} 

\end{document}
